\section{BACKGROUND AND RELATED WORK}
\label{sec:background}
\glsresetall

This chapter reviews the background and related work from academic and credible non-academic sources, such as, recognized international organizations, military institutions, and renowned think-tanks, in the areas of active and responsive cyber defence, cyber operations, cyber red teaming, and cyber red team oriented technical exercises.
Additionally, the gaps are identified and contribution to address these issues is emphasized.

\subsection{Responsive Cyber Defence and Initial Attribution}
\label{sec:rcd-work}
\glsresetall
Dewar \cite{Dewar2017} thoroughly explores the concepts of \gls{acd}, which is being adopted by a number of international actors, allowing to identify and stop increasingly occurring cyber incidents, as well as take offensive measures to minimize the attackers' capabilities. Dewar proposes variety of technical solutions to detect, analyse, identify and mitigate threats in real-time, such as, decoys, hacking back, \textit{``white worms''}, address hopping and honey pots. The research points out, that due to extraterritorial and aggressive nature of the \gls{acd} there are legal and political ramifications requiring special care when exerting this approach. Furthermore, publication implies that \gls{cno}, \gls{acd} and non-\gls{acd} actions are placed alongside and all fall under the cyber defence category, where all, except \gls{cno}, can be used both in peace- and war-time. \gls{cno} being exclusively limited to cyber warfare or military engagements employed either pre-emptively or deployed in advance of a kinetic manoeuvre. It is stated, that "ACD can be considered an approach to achieving cyber security predicated upon the deployment of measures to detect, analyze, identify and mitigate threats to and from communications systems and networks in real-time as well as the malicious actors involved." \cite{Dewar2017}
%In contrast to ACD, Fortified Cyber Defense (FCD)
%focusses on setting up defensive digital perimeters
%around key assets or potential targets.
%
%Resilient cyber defense (RCD) focusses on ensuring critical
%infrastructures and services which rely on digital networks
%continue to function in the event of a cyber-attack.
In a recent US Military Joint Publication \cite{USJCS2018} the new term ``Defensive Cyberspace Operation -- Responsive Activities'' is introduced to represent external cyberspace operations and defined as "[o]perations that are part of a defensive cyberspace operations mission that are taken external to the defended network or portion of cyberspace without the permission of the owner of the affected system". This definition overlaps with the concept of \gls{rcd}.
Additionally, Mansfield \cite{Mansfield2009} discusses the \gls{acd} and hacking the hackers from the perspective of botnet takeover, infiltration, controlled destruction and hacking back.
And Repik \cite{Repik2008} explores the techniques for dynamic network reconfiguration (e.g., address hopping) and decoys (e.g., honeypots, network telescopes) as the means of \gls{acd}.

From strategic and legal perspective, Denning \cite{Denning2014} looks at active and passive air and missile defence concept applicability to cyberspace. Denning defines the \gls{acd} as "direct defensive action taken to destroy, nullify, or reduce the effectiveness of cyber threats against friendly forces and assets" \cite{Denning2014} and is characterized by its scope of effects, degree of cooperation, types of effects, and degree of automation. Furthermore, \gls{acd} ethical and legal principles are addressed, such as, authority, third-party immunity, necessity, proportionality, human involvement, and civil liberties.
Bradbury \cite{Bradbury2013} discusses the principles of \gls{acd} and questions its necessity, legal and ethical considerations. Bradbury concludes, that ethical and legal considerations are shifting depending on who is executing \gls{acd}.
Furthermore, Schmitt \cite{Schmitt2002} considers seven factors of \gls{acd} -- severity, immediacy, directness, invasiveness, measurability, presumptive legitimacy, and responsibility. Focusing on effects-based approach that looks at the result of an attack, rather than what is used.

Publicly announced nation state initiatives, such as, US DARPA\footnote{\label{fnote:darpa}DARPA. "Active Cyber Defence (ACD) (Archived)". \url{https://www.darpa.mil/program/active-cyber-defense}. Accessed: 19/09/2018} has initiated an \gls{acd} project with a scope to develop a collection of synchronized, real-time capabilities to discover, define, analyse and mitigate cyber threats and vulnerabilities, enabling cyber defender readiness to disrupt and neutralize cyber attacks as they happen, however, these capabilities would be solely defensive in nature and specifically excludes research into cyber offence capabilities.
Also, US National Security Agency defines the \gls{acd} as "[e]nabling the real-time defense of critical national security networks by leading the integration, synchronization, and automation of cyber defense services and capabilities."\footnote{\label{fnote:nsa}NSA CSS. Active Cyber Defence (ACD). \url{https://www.iad.gov/iad/programs/iad-initiatives/active-cyber-defense.cfm}. Accessed 19/09/2018}
It is worth mentioning, the ADHD\footnote{\label{fnote:adhd}Active Defense Harbinger Distribution. \url{https://adhdproject.github.io/}. Accessed: 15/09/2018}, which is a GNU/Linux distribution collecting the most common tools and solutions to be used for \gls{acd}.
Additionally, UK's National Cyber Security Centre (NCSC)\footnote{UK GCHQ NCSC. Active Cyber Defense. \url{https://www.ncsc.gov.uk/active-cyber-defence}. Accessed: 16/10/2018} has started the Active Cyber Defense programme in addressing the cyber attacks in near real-time.

From a strictly theoretical perspective, Lu, Xu and Yi \cite{Lu2013} perform the mathematical modelling of \gls{acd} based on computer malware and biological epidemic game-theoretic models. In this study the authors acknowledge the \textit{``white worms''} as the only \gls{acd} measure. Lu et. al. determine, that defence based on \gls{ids}, firewalls and anti-malware tools is reactive and fundamentally asymmetrical. This can be eliminated by the use of \gls{acd}, offering strategic interaction between the attacker and the defender.

In the most significant identified work related to responsive cyber defence, Maybaum et.al. \cite{Maybaum2014} explore the concepts beyond the \gls{acd} and address the \gls{rcd} as a new approach to counter the increasing number of threats that \gls{it} systems have to face. The technical research paper assesses various tools and approaches which can be used for the \gls{rcd} not only from the technical perspective, but also taking into account the legal implications. Maybaum et.al. define \gls{rcd} as "[...] any activities conducted to defend one’s own IT systems or networks [...] against an on-going cyberattack by gaining access to, modifying or deleting data or services in other IT systems or networks [...] without the supposed or actual consent of their rightful owners or operators."
To complement it, Brangetto, Min\'{a}rik and Stinissen \cite{Brangetto2014-2} explore the legal implications of \gls{rcd} from the international and domestic law perspectives. Brangetto et.al. consider this approach only applicable to states and define it as "[...] the protection of a designated Communications and Information System (CIS) against an ongoing cyberattack by employing measures directed against the CIS from which the cyberattack originates, or against third-party CIS which are involved." Article elaborates, that \gls{rcd} can be seen as the subset for \gls{acd}, but \gls{rcd} is conducted only against actual and ongoing cyber attack, and it cannot be pre-emptive or retaliatory.

It is worth mentioning, an \gls{ai} technologies implemented for fighting cyber attacks in real-time by the \textit{Darktrace}\footnote{Darktrace. \url{https://www.darktrace.com/en/}. Accessed: 19/10/2018}. This approach implements the human immune system principles into an \gls{ai} driven solution to fight malware and cyber attacks originating from the external networks and malicious insiders. Darktrace elaborates, that nowadays, both the cyber attackers and defenders use the \gls{ai} technologies either to conduct the attacks or to deliver the defence with a higher levels of sophistication.


\textbf{System log file analysis.}
The vast topic of log analysis is tackled only from the perspective of unsupervised detection of anomalous messages from system logs as benefiting the cyber red team activities and does not constitute the entire literature.
Xu, Huang, Fox, Patterson and Jordan \cite{Xu2009} suggest an unsupervised method for anomalous event sequence detection using principal component analysis (PCA). By using source code analysis for detecting event type sequences, the vectors are derived with their attributes representing the number of events of a particular type in the sequence. Vectors with frequently occurring patterns are filtered out, assuming that they represent normal event sequences.
Oliner, Aiken and Stearley \cite{Oliner2008} propose an unsupervised \textit{Nodeinfo} algorithm, which analyses network node generated events from past days, divided into hourly frames (\textit{nodehours}), for anomaly detection. Shannon information entropy-based anomaly score is calculated for each nodehour depending on log file word occurrences in network nodes. Fewer occurrences would generate a higher anomaly score.
Du, Li, Zheng and Srikumar \cite{Du2017} propose the \textit{DeepLog} algorithm, which uses long short-term memory (LSTM) neural networks for anomaly detection in event type sequences, by predicting the event type probability from previous types in the sequence.
Yamanishi and Maruyama \cite{Yamanishi2005} suggest an unsupervised method for network failure prediction from system log error events. This is attempted by dividing the log into time-based sessions, which are modelled with hidden Markov mixture models, while model parameters are learned in an unsupervised manner. An anomaly score is calculated for each session and is considered as anomalous if its score exceeds a threshold.
In addition to aforementioned methods, a number of other approaches have been suggested for system log anomaly detection, including clustering \cite{Makanju2012}, invariant mining \cite{Lou2010}, and hybrid machine learning algorithms \cite{Liu2018}.

Additionally, It is worth mentioning an initial attempt for cyber red team infrastructure command and control server protection solution \textit{RedELK}\footnote{RedELK -- Red Team's SIEM. \url{https://github.com/outflanknl/RedELK}. Accessed: 19/10/2018}, which is aimed at implementing \textit{ELK Stack} to aggregate the data from the \gls{cnc} servers and maintain the visibility over them. This solution is focused on a small fraction of the cyber red team's operational infrastructure protection and could be applicable to countering the \gls{cnc} phase of the cyber attack kill-chain.


\textbf{Cyber deceptions.}
Cyber deception is recognized as an applicable set of techniques for the active cyber defence.
Provos \cite{Provos2004} proposes the virtual honeypot framework design and implementation for tracking malware and deceiving malicious activities. 
Wagener et.al. \cite{Wagener2011} and Zhang et.al. \cite{Zhang2017} identify the shortcomings of traditional honeypots and explore the adaptive and self-configurable honeypots allowing its behaviour adjustment according to the adversary actions and source of incoming connection.

Heckman et.al. \cite{Heckman2013} \cite{Heckman2015} admits that traditional approaches to cyber defence are inadequate and explores denial and deception (D\&D) as an option within \gls{acd}. Heckman proposes deceptions as part of a \gls{cdo}, such as, honeypots, honeyclients, honeytokens, and tarpits against an \gls{apt}. This paradigm of cyber denial and deception aims at influencing attacker in a way that gives the deceiver an advantage, by forcing adversary to move more slowly, expend more resources, and take greater risks.
The concepts were tested in a MITRE organized \gls{rt} versus \gls{bt} real-time war-game experiment, where a MITRE developed \textit{Blackjack} \gls{cnd} tool was employed for analysing adversary traffic by applying rules engine to enforce policy to each request. The exercise consisted of four teams (white, red, blue \gls{cnd}, and blue D\&D), where the \gls{bt} played the scenario of planning an attack against the enemy compound. Denial and deception techniques were found to be effective.
%Describes denial and deception techniques by using deception chain meta model: define purpose, collect intelligence, design cover story, plan, prepare, execute, monitor, and reinforce.
Geers \cite{Geers2010} proposes cyber attack deterrence by denial of capability, communication, and credibility.
F\`{a}rar, Bahsi and Blumbergs \cite{Farar2017} explore the use of cyber deceptions for network intrusion early warning, by focusing on stopping the attacks at their early stages of cyber kill-chain.
Cymmetria \textit{MazeRunner}\footnote{\label{fnote:mazerunner}Mazerunner. \url{https://cymmetria.com/}. Accessed: 15/09/2018} cyber deception solution delivers a novel approach by the use of digital \textit{breadcrumbs} to deceive the adversary leading to its detection and analysis.


\textbf{Initial Attribution.}
The active cyber defence employed methods, if successfully implemented can gather the required technical information allowing to perform the initial attribution and attempt the pursue of the adversary.
Rid \cite{Rid2015} introduces the \textit{Q-model} for cyber attack attribution assessment consisting of: concept (tactical/technical, operational, strategic), practice (asking right questions, targeting analysis, language indicators, modality of code, infrastructure, mistakes, stealth, \gls{cne}, stages of attack, geopolitical context, the form of damage inflicted), communication (releasing details on attack to boost the credibility, presenting more details to aid further attribution). Research proposes attributing offence to the offender to minimize the uncertainty at three levels: tactically, operationally, and strategically. Rid emphasizes that only a technical redesign of the Internet, consequently, could fully fix the attribution problem.
European Union Council \cite{EU2018} provides suggestions for decision-making regarding a joint EU diplomatic response to a malicious cyber activity, which requires attribution. The limited working paper presents the following steps: information collection, information assessment, political decision, and designing a response.
\subsection{Computer Network Operations}
\label{sec:cno-work}
\glsresetall
Robinson, Jones and Janicke \cite{Robinson2015} examine the concepts of cyber attack and cyber warfare. Robinson et.al. define cyber attack as "[a]n act in cyber space that could reasonably be expected to cause harm", and cyber warfare as "[t]he use of cyber attacks with a warfare-like intent". The researchers point out, that cyber war occurs only when nation states declare war and employ only cyber warfare. Research lists the following cyber warfare principles: lack of physical limitations, kinetic effects, stealth, mutability and inconsistency (e.g., the dynamic and changing nature of the cyberspace), identity and privileges (e.g., assumption of ones identity to gain the privileges), dual use (e.g., use of cyber weapons for defensive and offensive purposes), infrastructure control (e.g., control over the adversarial or third party systems to gain positional advantage), information as operational environment, attribution, defence and deterrence.
Leblanc et.al. \cite{Leblanc2011} assert that, per US Doctrine, \gls{cno} is comprised of \gls{cnd}, \gls{cna} and \gls{cne}. 
Additionally, US Joint Chiefs of Staff \cite{USJCS2014} acknowledge the concept of Information Operations as the integrated information-related capability employment within a military operation in concert with other performed operations to influence, disrupt, corrupt, or usurp the decision-making of existing and potential adversaries while protecting our own. Such operations integrate the employment of the core capabilities of electronic warfare, computer network operations, psychological operations, military deception and operations security.
Furthermore, US Military Joint Publication \cite{USJCS2018} is specifically addressed to explore the cyberspace operations from the perspective of their planning, coordination, responsibility division, execution, and assessment. This publication defines cyberspace operations as "[...] the employment of cyberspace capabilities where the primary purpose is to achieve objectives in or through cyberspace." This Joint Publication also redefines the terms used by the US military, such as, replacing \gls{cna} with ``offensive cyberspace operations'' and \gls{cne} with ``cyberspace exploitation''.
Schmitt et.al. \cite{Schmitt2017} define the cyber attack as "[...] a cyber operation, whether offensive or defensive, that is reasonably expected to cause injury or death to persons or damage or destruction to objects."
NATO Joint Publication \cite{NATOJP2018} promotes defensive cyberspace operations as one of the key protection challenges as the NATO's dependency on such systems is increasing.

Applegate \cite{Applegate2012} attempts to define the principle of manoeuvre, both in its offensive and defensive forms, within cyberspace as it relates to the traditional concept of manoeuvre in warfare. Applegate asserts, that the points of attack are moved in cyberspace instead of forces and defines the cyber manoeuvre as "[...] the application of force to capture, disrupt, deny, degrade, destroy or manipulate computing and information resources in order to achieve a position of advantage in respect to competitors". The main characteristics of a cyber manoeuvre are: speed, operational reach, access and control, dynamic evolution, stealth and limited attribution, rapid concentration, parallel and distributed nature.
Research introduces the following offensive cyber manoeuvre forms -- exploitive manoeuvre (capturing information resources to gain a strategic, operational or tactical competitive advantage), positional manoeuvre (capturing or compromising key physical or logical nodes in the information environment which can then be leveraged during follow-on operations), influencing manoeuvre (using cyber operations to get inside an adversary's decision cycle or even to force that decision cycle though direct or indirect actions). As well as presenting the defensive cyber manoeuvre forms -- perimeter defence \& defence in depth, moving target defence, deceptive defence, and counter attack. 

From a political and strategic stance, Mulvenon \cite{Mulvenon2009} explores the Chinese People's Liberation Army (PLA) conducted computer network operations from the perspective of the scenarios, doctrine, organization, and capabilities. Mulvenon states that defence and pre-emption are the core concepts of PLA and utilize offensive \gls{cno} as an attractive asymmetric weapon against the high-tech adversaries.

With a theoretical interest, Boukerche et.al. \cite{Boukerche2007} explore an agent-based and biological inspired real-time intrusion detection and security model for computer network operations. Research presents a novel intrusion detection model based on artificial immune and mobile agent paradigms for network intrusion detection and response.
\subsection{Cyber Red Teaming}
\label{sec:crt-work}
\glsresetall
From the publicly available authoritative sources, UK Ministry of Defence \cite{UK2013} recognizes increased use of red teaming as an alternative thinking and approach beneficial for defence. From this perspective, an independent \gls{rt} is formed to subject plans, programs, ideas and assumptions to rigorous analysis and challenge by applying a range of structured, creative and critical thinking techniques. \gls{rt} activities vary on the purpose and can include war-gaming, technical examination of vulnerabilities, system testing, or providing perspective through the eyes of adversary. Despite not focused on \gls{crt} directly, this doctrine grants valuable insight into \gls{rt} experience, assembly, leadership, tool-set, tasking, and phases of red teaming. Additionally, presents the golden rules of red teaming: timeliness (delivered in right time), quality (usefulness of the deliverables), and access (tailored for the end user comprehensiveness level). It is suggested, that optimum team size is generally considered between five and nine experts.
%Guidelines for good red teaming
%1. Plan red teaming from the outset. It cannot work as an afterthought.
%2. Create the right conditions. Red teaming needs an open, learning
%culture, accepting of challenge and criticism.
%3. Support the red team. Value and use its contribution to inform
%decisions and improve outcomes.
%4. Provide clear objectives.
%5. Fit the tool to the task. Select an appropriate team leader and follow
%their advice in the selection and employment of the red team.
%6. Promote a constructive approach which works towards overall success.
%7. Poorly conducted red teaming is pointless; do it well, do it properly.
%
%Steps for the end user
%Step 1 – Identify the specific task that they are to undertake
%Step 2 – Identify an appropriate red team leader and potential red team
%Step 3 – Task and empower the red team leader
%
%The red team leader should be
%allowed to run the team, employing the techniques that the leader feels
%are appropriate to the task.
%
%Critical and creative thinkers will form the core of the team.
%
%Phases of the red teaming:
%• Diagnostic phase. Is the information accurate, well-evidenced,
%logical and underpinned by valid assumptions?
%• Creative phase. Is the problem artificially constrained; have all
%possible options been considered; have the consequences been
%thought through?
%• Challenge phase. Are the options offered robust; are they
%resilient to shock, disruption or outright challenge; which of the
%options is the strongest; what are the chances of a successful
%outcome?
%
%Lists and describes the red teaming analytical techniques.
US Joint Force Development \cite{USJCS2016} looks at command red teaming with a purpose of decision support by providing an independent and skilled capability with critical and creative thought to fully explore alternatives in plans, operations, and intelligence analysis. \gls{rt} activities include decision support, critical review, adversary emulation, vulnerability testing, operation planning and operational design. US acknowledges, that cyberspace aggressors can be considered as a specialized and highly-focused red teams. Similar to UK perspective, this doctrine does not implicitly address \gls{crt}, but provides applicable aspects to adversary emulation and red cell operations.
%NATO calls this - "Alternative analysis"
%
%Red teams use a technique called adversary
%emulation to role play the mindset and decisions of an adversary, but
%they do not role play the full range of adversary actions as a red cell
%does.
%
%Red Cell. An element that simulates the strategic and tactical
%responses, including force employment and other objective factors, of
%a defined adversary.
%
%Adversary emulation involves simulating the behavioral responses of an
%adversary, actor, or stakeholder during an exercise, wargaming event, or an analytical
%effort, thus helping the staff understand the unique perceptions and mindset of the actor.
Additionally, Longbine \cite{Longbine2008} looks at the red teaming and acknowledges the tendency towards asymmetric warfare, where commanders require independent and skilled \gls{rt} to understand the battlefield and ultimately achieving success. Longbine explores various \gls{rt} definitions from the US perspective and highlights the core concepts -- decision-making, challenging the thinking, providing alternative analysis and perspective. Despite, not focused on cyber component, the monograph presents key ideas for \gls{rt} engagements, such as, threat emulation and conducting vulnerability assessment.
Furthermore, European central bank's developed unified framework for threat intelligence-based ethical red teaming (TIBER-EU) \cite{ECB2018} is focused at describing the controlled conduct of bespoke intelligence-led red teaming to mimic the \gls{ttp} of the existing advanced threat actors against the European Union's core financial infrastructure. The process, consisting of four main phases -- generic threat landscape, preparation, testing, and closure, attempts to cover a broad attack surface including elements, such as, people, processes, and technologies.

With a practical approach, Brangetto, \c{C}al\i\c{s}kan and R\~{o}igas \cite{Brangetto2015} look at military \gls{crt}, which mimics the mind-set and actions of the attacker to improve the security of one's own organization. \gls{crt} is defined as "[...] an element that conducts vulnerability assessments in a realistic threat environment and with an adversarial point of view, on specified information systems, in order to enhance an organisation’s level of security" \cite{Brangetto2015}. The study explores the \gls{crt} purpose, tasking, formation, structure, and legal implications. Additionally, technical exercises and cyber ranges are proposed for testing, evaluating and mimicking the adversarial actions for cyberspace concepts, policies and technologies.
Adkins \cite{Adkins2013} explores the option of cyber espionage and \gls{crt} to combat terrorism and its propaganda, recruitment, training, fundraising, communication and targeting. The \gls{crt} general principles are proposed for team assembly (penetration testers, social engineers, reverse engineers, \gls{ids} specialists, and language specialists), equipment (computers, servers, cellphones, tablet computers, penetration testing lab, and \gls{ids} solutions), and attack philosophy (avoidance of readily-available and downloadable solutions by the terrorists, limiting the attack spreading just to intended target, connection bouncing through proxy servers).

To expand these concepts, Yuen \cite{Yuen2015} explores automated \gls{crt} with automated planning by the use of \gls{ai}, such as, machine learning, hierarchical task networks, planning graphs, and state-space graphs to identify and plan the engagement as fast as possible in the applicable scope. Yuen acknowledges that \gls{crt} has a wider scope than penetration testing or vulnerability assessment, and is performed to appraise the infrastructure, processes, and personnel vulnerabilities to cyber attacks. 
Yuen, Randhawa, Turnbull, Hernandez and Dean \cite{Yuen2015-2} \cite{Randhawa2018} propose a \textit{Trogdor} system prototype for automated \gls{crt} based on state-of-the-art automated planners and \gls{ai} techniques to generate attack plans and model the organization's network at multiple levels of abstraction. Proposed system puts emphasis on visual analytics to achieve situational awareness over organization, intrusion paths, attack steps, attack patterns, attack quality and attack impact. It includes reconnaissance, network scanning, vulnerability analysis, penetration testing, attack graph generation, and risk/impact assessment. 
\subsection{Cyber Attack Kill Chain}
\label{sec:killchain-work}
\glsresetall
Hutchins, Cloppert and Amin \cite{Hutchins2011} explores the intelligence driven \gls{cnd} analysis of adversary, used \gls{tttp}, operation patterns and intention characteristics. Researchers present an intrusion kill-chain model to counter the evolving sophistication of cyber attacks. The kill-chain, in a systematic way, targets and engages the adversary at the phases of reconnaissance, weaponization, delivery, exploitation, installation, command and control, and execution of objectives. This approach attempts to detect, deny, disrupt, degrade, deceive or destroy the adversary and its attack at every stage of the kill-chain. %The most common delivery methods - "email attachments, websites, and USB removable media"
To further support this, Lockheed Martin \cite{LockheedMartin2015} acknowledges the significance of \gls{cii} for the national security and prosperity, and explores cyber kill-chain from the adversary and defender perspectives. To protect these infrastructures against an adversary, which is capable of defeating most common computer network defence mechanisms, every intrusion needs to be analysed to understand the adversarial motivations and strategies, allowing to break the attack chain with just one mitigation.
Malone proposes to expand the Hutchins et.al. developed cyber kill-chain \cite{Malone2016} by focusing on expanding the execution of objectives stage within the target information system. The proposed expansion adds two sub chains -- internal kill-chain (reconnaissance, exploitation, privilege escalation, lateral movement, target manipulation), and target manipulation kill-chain (reconnaissance, exploitation, weaponization, installation, execution).
Furthermore, Additional attack disruption methods have been proposed and adapted besides the Lockheed Martin's ``Cyber Kill Chain'' \cite{LockheedMartin2018}, such as, Mandiant/FireEye ``Attack Lifecycle Model'' \cite{FireEye2018}, Microsoft ``Attack Kill Chain'' \cite{Microsoft2018}, SANS ``The Industrial Control System Cyber Kill Chain'' \cite{SANS-ICS}, and MITRE ``ATT\&CK'' \cite{MITRE-ATTACK}.

Kim, Kwon and Kyu \cite{Kim2018} propose the modified cyber kill-chain model for multimedia service environments, such as, IoT. Kim et.al. study the limitation of the existing cyber kill-chain models (e.g. Lockheed Martin, FireEye, Command Five) and endorse the information security focus on detecting the actions happening within the network. Research proposes to expand the cyber kill-chain model with the internal network kill-chain stages -- internal reconnaissance, weaponization, delivery, exploitation, and installation.
Yadav and Rao \cite{Yadav2016} review the cyber kill-chain and present the popular attack tools and defensive options at every stage of a cyber attack.
Al-Mohannadi et.al. \cite{Al-Mohannadi2016} look at modelling the cyber attacks, which have not yet happened, by the use of diamond model, kill-chain and attack graphs.

Wen et.al. \cite{Wen2017} \cite{Wen2018} look at \gls{apt} detection as a measurable mathematical problem through Bayesian classification with correction factor. By studying the cyber kill-chain, a solution is proposed for \gls{apt} detection based on cyber security monitoring and intelligence gathering. The introduced approach utilizes information acquired from the online sources or acquired from real-time detection, and is divided into three categories -- attack intelligence (e.g., sensor logs, firewall/anti-virus alerts), attack behaviour (e.g., \textit{phishing}, \gls{dos}), attack events (e.g., attacker profiling, \gls{tttp}). Furthermore, tool prototype implementation is described based on \textit{JESS} -- the rule language engine for the \textit{Java}, but with no real testing and application presented in the paper.
Moskal, Yang and Kuhl \cite{Moskal2017} propose a cyber-based attacker behaviour modelling in conjunction with the ``Cyber Attack Scenario and Network Defense Simulator'' (CASCADES) to model the interaction between the network and the attackers. The approach simulates and measures the interaction between various generated types of attackers and network configurations under different scenarios. Attacker behaviour and decision-making process modelling is based on a reduced single attack action, which is executed against an integrated cyber kill-chain with fuzzy logic rules for each attack stage.

\subsection{Cyber Red Team Technical Exercises}
\label{sec:exercise-work}
\glsresetall
In a comprehensive manner, Leblanc et.al. \cite{Leblanc2011} explore and analyse multiple war-gaming exercises and implemented exercise support tool-sets.
ARENA provides cyber attack modelling, execution and simulated network construction, primary used to analyse \gls{ids}. Solution is capable to simulate simple attacks ranging from \gls{dos} to back-door installation.
RINSE (Real-time Immersive Network Simulation Environment) is a live simulation of large scale complex \gls{wan}, where various \gls{lan} operators have to defend against attacks. It is capable to simulate simple attacks, such as, \gls{dos}, \gls{ddos}, and computer worms.
SECUSIM is an application with a purpose to specify attack mechanisms, verify defence mechanisms and evaluate their consequences.
OPNET is a network simulator solution, which can be used also for cyber attack simulation aiding the analysis and design of communication networks, devices, protocols and applications.
SUNY provides modelling and examining network performances under \gls{dos} attacks.
NetENGINE is a cyber attack simulation tool allowing to model very large and complex computer networks to train \gls{it} staff in combating cyber attacks. Various generic cyber attacks are executed against simulated networks, as well as against the in game communication channels used by the players.
SIMTEX is a simulation infrastructure with various computer network attacks used for training.
CAAJED focuses on kinetic effects of cyber attacks in a war situation, where attacks are manually implemented and their effects controlled by the operators. 
IWAR is a network attacks and defence simulator implementing common trojan horses, vulnerability scanners, malware, \gls{dos} and brute-force attacks.
RMC is an isolated physical network for \gls{cno} education and training with an attacking team and training supervisors.
DARPA National Cyber Range is a project with an aim to simulate cyber attacks on computer networks and help develop strategies to defend against them.
CyberStorm is the US Department of Homeland Security developed exercise with the aim of examining readiness and response mechanisms to a simulated cyber event. Participation is strictly limited to Five Eyes alliance members.
Piranet is the French developed response plan and a simulation exercise of a major cyber attack against France's \gls{cii}.
Divine Matrix is the India's war-gaming exercise to simulate a nuclear attack accompanied by a massive cyber attack with kinetic effects against India.

Mauer, Stackpole and Johnson \cite{Mauer2012} look at developing small team-based cyber security exercises for use at the university as a practical hands-on part within the courses. The research explains the management and roles of the engaged parties in the exercise creation and execution. The developed game network is comprised of small set of virtualized machines used by a group of participants to attack, defend and monitor the event.
DeLooze, McKeen, Mostow and Graig \cite{DeLooze2004} examine the US Strategic Command developed simulation environment to train and exercise \gls{cno} and determine if these complex concepts can be more effectively taught in the classroom. The simulation environment consists of ``Virtual Network Simulator'', comprised of two or more networked computers designed to represent attack effects in an interactive graphical environment, and the ``Internet Attack Simulator'', presenting a set of simple attacks, ranging from reconnaissance to \gls{dos}, available for launching against the network simulator's virtual network. Researchers confirm the benefit of \gls{cno} simulation exercises by measuring the increase of knowledge of the participants.

\subsection{Identified Gaps}
\glsresetall
Related work analysis shows that no substantial work has been done in the area of exploring the aspects the cyber red team technical capability development, execution of responsive cyber defence operations, and conducting technical exercises aimed at cyber red team training and skill development.
The main drawbacks are the following:
\begin{enumerate}
    \item red teaming is acknowledged by nations primary as a supportive element to decision making process, providing an alternative analysis on various topics, but not anticipated as an operational capability to actively pursue the commander assigned external targets. The red team, consisting of highly skilled experts, might be engaged in cyber espionage, or targeting organization's own defences to provide adversarial perspective. However, it is not yet accepted to be used actively against any other target or threat actor. A broader perspective and the scope of cyber red team capability applicability has to be expanded;
    \item cyber kill-chain has been thoroughly reviewed from the defence perspective, as its primary area of application, however, the view from the adversarial aspect on countering the cyber kill-chain should be explored. By countering the cyber kill-chain, in this work, is understood the red team \gls{tttp} allowing to bypass the detection and prevention mechanisms at every phase of the kill-chain. This is especially important when planning the cyber operations against an advanced adversary to ensure the maximum possible success rate;
    \item cyber red team technical capabilities and techniques for tool development to be used against a real adversary are not explored fully. The proper techniques and tools are even more important when a small cyber red team (up to 9 members) is considered to be engaged in the computer network operations;
    \item the aspects of active cyber defence are explored, but responsive activities are not considered. Despite the responsive cyber defence being a subset for active cyber defence, it possesses distinctive differences, especially that of responsive defence being directed primarily at external networks and systems instead of mainly focusing on activities within own systems;
    \item various sources interpret computer network operations differently and present various definitions, ranging from any attacks executed over computer networks up to being only applicable to the cyber warfare. A coherent definition is required to have a solid understanding of such operation applicability and implied characteristics;
    \item no special guide, manuals or proposals exist on computer network operation execution by the cyber red team. It might be expected, that such guidelines would be developed for nation's internal use only and not shared publicly, however, there is a lack of publicly available information on this topic; and
    \item the tackled issues for the cyber defence exercises are either narrow in scope, specific to a particular nation or small set of nations, restricted only to exercising just the decision making process or a small subset of a full-scale cyber operation, or limited to just simulation of common cyber attacks. Additionally, majority of the exercises are delivered for the \textit{Blue Team} defensive capability building, and the \textit{Red Team} is either simulated or role-playing the adversary. However, a dedicated technical exercise for training cyber red team capabilities is required.
\end{enumerate}

This thesis addresses the identified gaps to introduce
a unified understanding and definitions for the core principles, such as, \gls{rcd}, \gls{cno}, \gls{crt} and ``cyber attack kill-chain'',
explores the cyber red team offensive and defensive capabilities from the perspective of the cyber red team responsive computer network operations, 
presents practical and immediately applicable cyber red team \gls{tttp} to complement the conducted responsive operations,
proposes automated anomaly detection and cyber deception usage within the cyber red team's operational network for its protection, situational awareness, threat assessment and adversary tracking, and
defines the requirements for the technical exercises aimed directly at cyber red team capability development and preparation for real-life full-spectrum cyber operation execution.

\subsection{Responsive Cyber Defence and Initial Attribution}
\label{sec:rcd-work}
\glsresetall
Dewar \cite{Dewar2017} thoroughly explores the concepts of \gls{acd}, which is being adopted by a number of international actors, allowing to identify and stop increasingly occurring cyber incidents, as well as take offensive measures to minimize the attackers' capabilities. Dewar proposes variety of technical solutions to detect, analyse, identify and mitigate threats in real-time, such as, decoys, hacking back, \textit{``white worms''}, address hopping and honey pots. The research points out, that due to extraterritorial and aggressive nature of the \gls{acd} there are legal and political ramifications requiring special care when exerting this approach. Furthermore, publication implies that \gls{cno}, \gls{acd} and non-\gls{acd} actions are placed alongside and all fall under the cyber defence category, where all, except \gls{cno}, can be used both in peace- and war-time. \gls{cno} being exclusively limited to cyber warfare or military engagements employed either pre-emptively or deployed in advance of a kinetic manoeuvre. It is stated, that "ACD can be considered an approach to achieving cyber security predicated upon the deployment of measures to detect, analyze, identify and mitigate threats to and from communications systems and networks in real-time as well as the malicious actors involved." \cite{Dewar2017}
%In contrast to ACD, Fortified Cyber Defense (FCD)
%focusses on setting up defensive digital perimeters
%around key assets or potential targets.
%
%Resilient cyber defense (RCD) focusses on ensuring critical
%infrastructures and services which rely on digital networks
%continue to function in the event of a cyber-attack.
In a recent US Military Joint Publication \cite{USJCS2018} the new term ``Defensive Cyberspace Operation -- Responsive Activities'' is introduced to represent external cyberspace operations and defined as "[o]perations that are part of a defensive cyberspace operations mission that are taken external to the defended network or portion of cyberspace without the permission of the owner of the affected system". This definition overlaps with the concept of \gls{rcd}.
Additionally, Mansfield \cite{Mansfield2009} discusses the \gls{acd} and hacking the hackers from the perspective of botnet takeover, infiltration, controlled destruction and hacking back.
And Repik \cite{Repik2008} explores the techniques for dynamic network reconfiguration (e.g., address hopping) and decoys (e.g., honeypots, network telescopes) as the means of \gls{acd}.

From strategic and legal perspective, Denning \cite{Denning2014} looks at active and passive air and missile defence concept applicability to cyberspace. Denning defines the \gls{acd} as "direct defensive action taken to destroy, nullify, or reduce the effectiveness of cyber threats against friendly forces and assets" \cite{Denning2014} and is characterized by its scope of effects, degree of cooperation, types of effects, and degree of automation. Furthermore, \gls{acd} ethical and legal principles are addressed, such as, authority, third-party immunity, necessity, proportionality, human involvement, and civil liberties.
Bradbury \cite{Bradbury2013} discusses the principles of \gls{acd} and questions its necessity, legal and ethical considerations. Bradbury concludes, that ethical and legal considerations are shifting depending on who is executing \gls{acd}.
Furthermore, Schmitt \cite{Schmitt2002} considers seven factors of \gls{acd} -- severity, immediacy, directness, invasiveness, measurability, presumptive legitimacy, and responsibility. Focusing on effects-based approach that looks at the result of an attack, rather than what is used.

Publicly announced nation state initiatives, such as, US DARPA\footnote{\label{fnote:darpa}DARPA. "Active Cyber Defence (ACD) (Archived)". \url{https://www.darpa.mil/program/active-cyber-defense}. Accessed: 19/09/2018} has initiated an \gls{acd} project with a scope to develop a collection of synchronized, real-time capabilities to discover, define, analyse and mitigate cyber threats and vulnerabilities, enabling cyber defender readiness to disrupt and neutralize cyber attacks as they happen, however, these capabilities would be solely defensive in nature and specifically excludes research into cyber offence capabilities.
Also, US National Security Agency defines the \gls{acd} as "[e]nabling the real-time defense of critical national security networks by leading the integration, synchronization, and automation of cyber defense services and capabilities."\footnote{\label{fnote:nsa}NSA CSS. Active Cyber Defence (ACD). \url{https://www.iad.gov/iad/programs/iad-initiatives/active-cyber-defense.cfm}. Accessed 19/09/2018}
It is worth mentioning, the ADHD\footnote{\label{fnote:adhd}Active Defense Harbinger Distribution. \url{https://adhdproject.github.io/}. Accessed: 15/09/2018}, which is a GNU/Linux distribution collecting the most common tools and solutions to be used for \gls{acd}.
Additionally, UK's National Cyber Security Centre (NCSC)\footnote{UK GCHQ NCSC. Active Cyber Defense. \url{https://www.ncsc.gov.uk/active-cyber-defence}. Accessed: 16/10/2018} has started the Active Cyber Defense programme in addressing the cyber attacks in near real-time.

From a strictly theoretical perspective, Lu, Xu and Yi \cite{Lu2013} perform the mathematical modelling of \gls{acd} based on computer malware and biological epidemic game-theoretic models. In this study the authors acknowledge the \textit{``white worms''} as the only \gls{acd} measure. Lu et. al. determine, that defence based on \gls{ids}, firewalls and anti-malware tools is reactive and fundamentally asymmetrical. This can be eliminated by the use of \gls{acd}, offering strategic interaction between the attacker and the defender.

In the most significant identified work related to responsive cyber defence, Maybaum et.al. \cite{Maybaum2014} explore the concepts beyond the \gls{acd} and address the \gls{rcd} as a new approach to counter the increasing number of threats that \gls{it} systems have to face. The technical research paper assesses various tools and approaches which can be used for the \gls{rcd} not only from the technical perspective, but also taking into account the legal implications. Maybaum et.al. define \gls{rcd} as "[...] any activities conducted to defend one’s own IT systems or networks [...] against an on-going cyberattack by gaining access to, modifying or deleting data or services in other IT systems or networks [...] without the supposed or actual consent of their rightful owners or operators."
To complement it, Brangetto, Min\'{a}rik and Stinissen \cite{Brangetto2014-2} explore the legal implications of \gls{rcd} from the international and domestic law perspectives. Brangetto et.al. consider this approach only applicable to states and define it as "[...] the protection of a designated Communications and Information System (CIS) against an ongoing cyberattack by employing measures directed against the CIS from which the cyberattack originates, or against third-party CIS which are involved." Article elaborates, that \gls{rcd} can be seen as the subset for \gls{acd}, but \gls{rcd} is conducted only against actual and ongoing cyber attack, and it cannot be pre-emptive or retaliatory.

It is worth mentioning, an \gls{ai} technologies implemented for fighting cyber attacks in real-time by the \textit{Darktrace}\footnote{Darktrace. \url{https://www.darktrace.com/en/}. Accessed: 19/10/2018}. This approach implements the human immune system principles into an \gls{ai} driven solution to fight malware and cyber attacks originating from the external networks and malicious insiders. Darktrace elaborates, that nowadays, both the cyber attackers and defenders use the \gls{ai} technologies either to conduct the attacks or to deliver the defence with a higher levels of sophistication.


\textbf{System log file analysis.}
The vast topic of log analysis is tackled only from the perspective of unsupervised detection of anomalous messages from system logs as benefiting the cyber red team activities and does not constitute the entire literature.
Xu, Huang, Fox, Patterson and Jordan \cite{Xu2009} suggest an unsupervised method for anomalous event sequence detection using principal component analysis (PCA). By using source code analysis for detecting event type sequences, the vectors are derived with their attributes representing the number of events of a particular type in the sequence. Vectors with frequently occurring patterns are filtered out, assuming that they represent normal event sequences.
Oliner, Aiken and Stearley \cite{Oliner2008} propose an unsupervised \textit{Nodeinfo} algorithm, which analyses network node generated events from past days, divided into hourly frames (\textit{nodehours}), for anomaly detection. Shannon information entropy-based anomaly score is calculated for each nodehour depending on log file word occurrences in network nodes. Fewer occurrences would generate a higher anomaly score.
Du, Li, Zheng and Srikumar \cite{Du2017} propose the \textit{DeepLog} algorithm, which uses long short-term memory (LSTM) neural networks for anomaly detection in event type sequences, by predicting the event type probability from previous types in the sequence.
Yamanishi and Maruyama \cite{Yamanishi2005} suggest an unsupervised method for network failure prediction from system log error events. This is attempted by dividing the log into time-based sessions, which are modelled with hidden Markov mixture models, while model parameters are learned in an unsupervised manner. An anomaly score is calculated for each session and is considered as anomalous if its score exceeds a threshold.
In addition to aforementioned methods, a number of other approaches have been suggested for system log anomaly detection, including clustering \cite{Makanju2012}, invariant mining \cite{Lou2010}, and hybrid machine learning algorithms \cite{Liu2018}.

Additionally, It is worth mentioning an initial attempt for cyber red team infrastructure command and control server protection solution \textit{RedELK}\footnote{RedELK -- Red Team's SIEM. \url{https://github.com/outflanknl/RedELK}. Accessed: 19/10/2018}, which is aimed at implementing \textit{ELK Stack} to aggregate the data from the \gls{cnc} servers and maintain the visibility over them. This solution is focused on a small fraction of the cyber red team's operational infrastructure protection and could be applicable to countering the \gls{cnc} phase of the cyber attack kill-chain.


\textbf{Cyber deceptions.}
Cyber deception is recognized as an applicable set of techniques for the active cyber defence.
Provos \cite{Provos2004} proposes the virtual honeypot framework design and implementation for tracking malware and deceiving malicious activities. 
Wagener et.al. \cite{Wagener2011} and Zhang et.al. \cite{Zhang2017} identify the shortcomings of traditional honeypots and explore the adaptive and self-configurable honeypots allowing its behaviour adjustment according to the adversary actions and source of incoming connection.

Heckman et.al. \cite{Heckman2013} \cite{Heckman2015} admits that traditional approaches to cyber defence are inadequate and explores denial and deception (D\&D) as an option within \gls{acd}. Heckman proposes deceptions as part of a \gls{cdo}, such as, honeypots, honeyclients, honeytokens, and tarpits against an \gls{apt}. This paradigm of cyber denial and deception aims at influencing attacker in a way that gives the deceiver an advantage, by forcing adversary to move more slowly, expend more resources, and take greater risks.
The concepts were tested in a MITRE organized \gls{rt} versus \gls{bt} real-time war-game experiment, where a MITRE developed \textit{Blackjack} \gls{cnd} tool was employed for analysing adversary traffic by applying rules engine to enforce policy to each request. The exercise consisted of four teams (white, red, blue \gls{cnd}, and blue D\&D), where the \gls{bt} played the scenario of planning an attack against the enemy compound. Denial and deception techniques were found to be effective.
%Describes denial and deception techniques by using deception chain meta model: define purpose, collect intelligence, design cover story, plan, prepare, execute, monitor, and reinforce.
Geers \cite{Geers2010} proposes cyber attack deterrence by denial of capability, communication, and credibility.
F\`{a}rar, Bahsi and Blumbergs \cite{Farar2017} explore the use of cyber deceptions for network intrusion early warning, by focusing on stopping the attacks at their early stages of cyber kill-chain.
Cymmetria \textit{MazeRunner}\footnote{\label{fnote:mazerunner}Mazerunner. \url{https://cymmetria.com/}. Accessed: 15/09/2018} cyber deception solution delivers a novel approach by the use of digital \textit{breadcrumbs} to deceive the adversary leading to its detection and analysis.


\textbf{Initial Attribution.}
The active cyber defence employed methods, if successfully implemented can gather the required technical information allowing to perform the initial attribution and attempt the pursue of the adversary.
Rid \cite{Rid2015} introduces the \textit{Q-model} for cyber attack attribution assessment consisting of: concept (tactical/technical, operational, strategic), practice (asking right questions, targeting analysis, language indicators, modality of code, infrastructure, mistakes, stealth, \gls{cne}, stages of attack, geopolitical context, the form of damage inflicted), communication (releasing details on attack to boost the credibility, presenting more details to aid further attribution). Research proposes attributing offence to the offender to minimize the uncertainty at three levels: tactically, operationally, and strategically. Rid emphasizes that only a technical redesign of the Internet, consequently, could fully fix the attribution problem.
European Union Council \cite{EU2018} provides suggestions for decision-making regarding a joint EU diplomatic response to a malicious cyber activity, which requires attribution. The limited working paper presents the following steps: information collection, information assessment, political decision, and designing a response.
\subsection{Cyber Attack Kill Chain}
\label{sec:killchain-work}
\glsresetall
Hutchins, Cloppert and Amin \cite{Hutchins2011} explores the intelligence driven \gls{cnd} analysis of adversary, used \gls{tttp}, operation patterns and intention characteristics. Researchers present an intrusion kill-chain model to counter the evolving sophistication of cyber attacks. The kill-chain, in a systematic way, targets and engages the adversary at the phases of reconnaissance, weaponization, delivery, exploitation, installation, command and control, and execution of objectives. This approach attempts to detect, deny, disrupt, degrade, deceive or destroy the adversary and its attack at every stage of the kill-chain. %The most common delivery methods - "email attachments, websites, and USB removable media"
To further support this, Lockheed Martin \cite{LockheedMartin2015} acknowledges the significance of \gls{cii} for the national security and prosperity, and explores cyber kill-chain from the adversary and defender perspectives. To protect these infrastructures against an adversary, which is capable of defeating most common computer network defence mechanisms, every intrusion needs to be analysed to understand the adversarial motivations and strategies, allowing to break the attack chain with just one mitigation.
Malone proposes to expand the Hutchins et.al. developed cyber kill-chain \cite{Malone2016} by focusing on expanding the execution of objectives stage within the target information system. The proposed expansion adds two sub chains -- internal kill-chain (reconnaissance, exploitation, privilege escalation, lateral movement, target manipulation), and target manipulation kill-chain (reconnaissance, exploitation, weaponization, installation, execution).
Furthermore, Additional attack disruption methods have been proposed and adapted besides the Lockheed Martin's ``Cyber Kill Chain'' \cite{LockheedMartin2018}, such as, Mandiant/FireEye ``Attack Lifecycle Model'' \cite{FireEye2018}, Microsoft ``Attack Kill Chain'' \cite{Microsoft2018}, SANS ``The Industrial Control System Cyber Kill Chain'' \cite{SANS-ICS}, and MITRE ``ATT\&CK'' \cite{MITRE-ATTACK}.

Kim, Kwon and Kyu \cite{Kim2018} propose the modified cyber kill-chain model for multimedia service environments, such as, IoT. Kim et.al. study the limitation of the existing cyber kill-chain models (e.g. Lockheed Martin, FireEye, Command Five) and endorse the information security focus on detecting the actions happening within the network. Research proposes to expand the cyber kill-chain model with the internal network kill-chain stages -- internal reconnaissance, weaponization, delivery, exploitation, and installation.
Yadav and Rao \cite{Yadav2016} review the cyber kill-chain and present the popular attack tools and defensive options at every stage of a cyber attack.
Al-Mohannadi et.al. \cite{Al-Mohannadi2016} look at modelling the cyber attacks, which have not yet happened, by the use of diamond model, kill-chain and attack graphs.

Wen et.al. \cite{Wen2017} \cite{Wen2018} look at \gls{apt} detection as a measurable mathematical problem through Bayesian classification with correction factor. By studying the cyber kill-chain, a solution is proposed for \gls{apt} detection based on cyber security monitoring and intelligence gathering. The introduced approach utilizes information acquired from the online sources or acquired from real-time detection, and is divided into three categories -- attack intelligence (e.g., sensor logs, firewall/anti-virus alerts), attack behaviour (e.g., \textit{phishing}, \gls{dos}), attack events (e.g., attacker profiling, \gls{tttp}). Furthermore, tool prototype implementation is described based on \textit{JESS} -- the rule language engine for the \textit{Java}, but with no real testing and application presented in the paper.
Moskal, Yang and Kuhl \cite{Moskal2017} propose a cyber-based attacker behaviour modelling in conjunction with the ``Cyber Attack Scenario and Network Defense Simulator'' (CASCADES) to model the interaction between the network and the attackers. The approach simulates and measures the interaction between various generated types of attackers and network configurations under different scenarios. Attacker behaviour and decision-making process modelling is based on a reduced single attack action, which is executed against an integrated cyber kill-chain with fuzzy logic rules for each attack stage.

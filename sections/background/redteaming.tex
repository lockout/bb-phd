\subsection{Cyber Red Teaming}
\label{sec:crt-work}
\glsresetall
From the publicly available authoritative sources, UK Ministry of Defence \cite{UK2013} recognizes increased use of red teaming as an alternative thinking and approach beneficial for defence. From this perspective, an independent \gls{rt} is formed to subject plans, programs, ideas and assumptions to rigorous analysis and challenge by applying a range of structured, creative and critical thinking techniques. \gls{rt} activities vary on the purpose and can include war-gaming, technical examination of vulnerabilities, system testing, or providing perspective through the eyes of adversary. Despite not focused on \gls{crt} directly, this doctrine grants valuable insight into \gls{rt} experience, assembly, leadership, tool-set, tasking, and phases of red teaming. Additionally, presents the golden rules of red teaming: timeliness (delivered in right time), quality (usefulness of the deliverables), and access (tailored for the end user comprehensiveness level). It is suggested, that optimum team size is generally considered between five and nine experts.
%Guidelines for good red teaming
%1. Plan red teaming from the outset. It cannot work as an afterthought.
%2. Create the right conditions. Red teaming needs an open, learning
%culture, accepting of challenge and criticism.
%3. Support the red team. Value and use its contribution to inform
%decisions and improve outcomes.
%4. Provide clear objectives.
%5. Fit the tool to the task. Select an appropriate team leader and follow
%their advice in the selection and employment of the red team.
%6. Promote a constructive approach which works towards overall success.
%7. Poorly conducted red teaming is pointless; do it well, do it properly.
%
%Steps for the end user
%Step 1 – Identify the specific task that they are to undertake
%Step 2 – Identify an appropriate red team leader and potential red team
%Step 3 – Task and empower the red team leader
%
%The red team leader should be
%allowed to run the team, employing the techniques that the leader feels
%are appropriate to the task.
%
%Critical and creative thinkers will form the core of the team.
%
%Phases of the red teaming:
%• Diagnostic phase. Is the information accurate, well-evidenced,
%logical and underpinned by valid assumptions?
%• Creative phase. Is the problem artificially constrained; have all
%possible options been considered; have the consequences been
%thought through?
%• Challenge phase. Are the options offered robust; are they
%resilient to shock, disruption or outright challenge; which of the
%options is the strongest; what are the chances of a successful
%outcome?
%
%Lists and describes the red teaming analytical techniques.
US Joint Force Development \cite{USJCS2016} looks at command red teaming with a purpose of decision support by providing an independent and skilled capability with critical and creative thought to fully explore alternatives in plans, operations, and intelligence analysis. \gls{rt} activities include decision support, critical review, adversary emulation, vulnerability testing, operation planning and operational design. US acknowledges, that cyberspace aggressors can be considered as a specialized and highly-focused red teams. Similar to UK perspective, this doctrine does not implicitly address \gls{crt}, but provides applicable aspects to adversary emulation and red cell operations.
%NATO calls this - "Alternative analysis"
%
%Red teams use a technique called adversary
%emulation to role play the mindset and decisions of an adversary, but
%they do not role play the full range of adversary actions as a red cell
%does.
%
%Red Cell. An element that simulates the strategic and tactical
%responses, including force employment and other objective factors, of
%a defined adversary.
%
%Adversary emulation involves simulating the behavioral responses of an
%adversary, actor, or stakeholder during an exercise, wargaming event, or an analytical
%effort, thus helping the staff understand the unique perceptions and mindset of the actor.
Additionally, Longbine \cite{Longbine2008} looks at the red teaming and acknowledges the tendency towards asymmetric warfare, where commanders require independent and skilled \gls{rt} to understand the battlefield and ultimately achieving success. Longbine explores various \gls{rt} definitions from the US perspective and highlights the core concepts -- decision-making, challenging the thinking, providing alternative analysis and perspective. Despite, not focused on cyber component, the monograph presents key ideas for \gls{rt} engagements, such as, threat emulation and conducting vulnerability assessment.
Furthermore, European central bank's developed unified framework for threat intelligence-based ethical red teaming (TIBER-EU) \cite{ECB2018} is focused at describing the controlled conduct of bespoke intelligence-led red teaming to mimic the \gls{ttp} of the existing advanced threat actors against the European Union's core financial infrastructure. The process, consisting of four main phases -- generic threat landscape, preparation, testing, and closure, attempts to cover a broad attack surface including elements, such as, people, processes, and technologies.

With a practical approach, Brangetto, \c{C}al\i\c{s}kan and R\~{o}igas \cite{Brangetto2015} look at military \gls{crt}, which mimics the mind-set and actions of the attacker to improve the security of one's own organization. \gls{crt} is defined as "[...] an element that conducts vulnerability assessments in a realistic threat environment and with an adversarial point of view, on specified information systems, in order to enhance an organisation’s level of security" \cite{Brangetto2015}. The study explores the \gls{crt} purpose, tasking, formation, structure, and legal implications. Additionally, technical exercises and cyber ranges are proposed for testing, evaluating and mimicking the adversarial actions for cyberspace concepts, policies and technologies.
Adkins \cite{Adkins2013} explores the option of cyber espionage and \gls{crt} to combat terrorism and its propaganda, recruitment, training, fundraising, communication and targeting. The \gls{crt} general principles are proposed for team assembly (penetration testers, social engineers, reverse engineers, \gls{ids} specialists, and language specialists), equipment (computers, servers, cellphones, tablet computers, penetration testing lab, and \gls{ids} solutions), and attack philosophy (avoidance of readily-available and downloadable solutions by the terrorists, limiting the attack spreading just to intended target, connection bouncing through proxy servers).

To expand these concepts, Yuen \cite{Yuen2015} explores automated \gls{crt} with automated planning by the use of \gls{ai}, such as, machine learning, hierarchical task networks, planning graphs, and state-space graphs to identify and plan the engagement as fast as possible in the applicable scope. Yuen acknowledges that \gls{crt} has a wider scope than penetration testing or vulnerability assessment, and is performed to appraise the infrastructure, processes, and personnel vulnerabilities to cyber attacks. 
Yuen, Randhawa, Turnbull, Hernandez and Dean \cite{Yuen2015-2} \cite{Randhawa2018} propose a \textit{Trogdor} system prototype for automated \gls{crt} based on state-of-the-art automated planners and \gls{ai} techniques to generate attack plans and model the organization's network at multiple levels of abstraction. Proposed system puts emphasis on visual analytics to achieve situational awareness over organization, intrusion paths, attack steps, attack patterns, attack quality and attack impact. It includes reconnaissance, network scanning, vulnerability analysis, penetration testing, attack graph generation, and risk/impact assessment. 
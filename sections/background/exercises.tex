\subsection{Cyber Red Team Technical Exercises}
\label{sec:exercise-work}
\glsresetall
In a comprehensive manner, Leblanc et.al. \cite{Leblanc2011} explore and analyse multiple war-gaming exercises and implemented exercise support tool-sets.
ARENA provides cyber attack modelling, execution and simulated network construction, primary used to analyse \gls{ids}. Solution is capable to simulate simple attacks ranging from \gls{dos} to back-door installation.
RINSE (Real-time Immersive Network Simulation Environment) is a live simulation of large scale complex \gls{wan}, where various \gls{lan} operators have to defend against attacks. It is capable to simulate simple attacks, such as, \gls{dos}, \gls{ddos}, and computer worms.
SECUSIM is an application with a purpose to specify attack mechanisms, verify defence mechanisms and evaluate their consequences.
OPNET is a network simulator solution, which can be used also for cyber attack simulation aiding the analysis and design of communication networks, devices, protocols and applications.
SUNY provides modelling and examining network performances under \gls{dos} attacks.
NetENGINE is a cyber attack simulation tool allowing to model very large and complex computer networks to train \gls{it} staff in combating cyber attacks. Various generic cyber attacks are executed against simulated networks, as well as against the in game communication channels used by the players.
SIMTEX is a simulation infrastructure with various computer network attacks used for training.
CAAJED focuses on kinetic effects of cyber attacks in a war situation, where attacks are manually implemented and their effects controlled by the operators. 
IWAR is a network attacks and defence simulator implementing common trojan horses, vulnerability scanners, malware, \gls{dos} and brute-force attacks.
RMC is an isolated physical network for \gls{cno} education and training with an attacking team and training supervisors.
DARPA National Cyber Range is a project with an aim to simulate cyber attacks on computer networks and help develop strategies to defend against them.
CyberStorm is the US Department of Homeland Security developed exercise with the aim of examining readiness and response mechanisms to a simulated cyber event. Participation is strictly limited to Five Eyes alliance members.
Piranet is the French developed response plan and a simulation exercise of a major cyber attack against France's \gls{cii}.
Divine Matrix is the India's war-gaming exercise to simulate a nuclear attack accompanied by a massive cyber attack with kinetic effects against India.

Mauer, Stackpole and Johnson \cite{Mauer2012} look at developing small team-based cyber security exercises for use at the university as a practical hands-on part within the courses. The research explains the management and roles of the engaged parties in the exercise creation and execution. The developed game network is comprised of small set of virtualized machines used by a group of participants to attack, defend and monitor the event.
DeLooze, McKeen, Mostow and Graig \cite{DeLooze2004} examine the US Strategic Command developed simulation environment to train and exercise \gls{cno} and determine if these complex concepts can be more effectively taught in the classroom. The simulation environment consists of ``Virtual Network Simulator'', comprised of two or more networked computers designed to represent attack effects in an interactive graphical environment, and the ``Internet Attack Simulator'', presenting a set of simple attacks, ranging from reconnaissance to \gls{dos}, available for launching against the network simulator's virtual network. Researchers confirm the benefit of \gls{cno} simulation exercises by measuring the increase of knowledge of the participants.
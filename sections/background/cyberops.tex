\subsection{Computer Network Operations}
\label{sec:cno-work}
\glsresetall
Robinson, Jones and Janicke \cite{Robinson2015} examine the concepts of cyber attack and cyber warfare. Robinson et.al. define cyber attack as "[a]n act in cyber space that could reasonably be expected to cause harm", and cyber warfare as "[t]he use of cyber attacks with a warfare-like intent". The researchers point out, that cyber war occurs only when nation states declare war and employ only cyber warfare. Research lists the following cyber warfare principles: lack of physical limitations, kinetic effects, stealth, mutability and inconsistency (e.g., the dynamic and changing nature of the cyberspace), identity and privileges (e.g., assumption of ones identity to gain the privileges), dual use (e.g., use of cyber weapons for defensive and offensive purposes), infrastructure control (e.g., control over the adversarial or third party systems to gain positional advantage), information as operational environment, attribution, defence and deterrence.
Leblanc et.al. \cite{Leblanc2011} assert that, per US Doctrine, \gls{cno} is comprised of \gls{cnd}, \gls{cna} and \gls{cne}. 
Additionally, US Joint Chiefs of Staff \cite{USJCS2014} acknowledge the concept of Information Operations as the integrated information-related capability employment within a military operation in concert with other performed operations to influence, disrupt, corrupt, or usurp the decision-making of existing and potential adversaries while protecting our own. Such operations integrate the employment of the core capabilities of electronic warfare, computer network operations, psychological operations, military deception and operations security.
Furthermore, US Military Joint Publication \cite{USJCS2018} is specifically addressed to explore the cyberspace operations from the perspective of their planning, coordination, responsibility division, execution, and assessment. This publication defines cyberspace operations as "[...] the employment of cyberspace capabilities where the primary purpose is to achieve objectives in or through cyberspace." This Joint Publication also redefines the terms used by the US military, such as, replacing \gls{cna} with ``offensive cyberspace operations'' and \gls{cne} with ``cyberspace exploitation''.
Schmitt et.al. \cite{Schmitt2017} define the cyber attack as "[...] a cyber operation, whether offensive or defensive, that is reasonably expected to cause injury or death to persons or damage or destruction to objects."
NATO Joint Publication \cite{NATOJP2018} promotes defensive cyberspace operations as one of the key protection challenges as the NATO's dependency on such systems is increasing.

Applegate \cite{Applegate2012} attempts to define the principle of manoeuvre, both in its offensive and defensive forms, within cyberspace as it relates to the traditional concept of manoeuvre in warfare. Applegate asserts, that the points of attack are moved in cyberspace instead of forces and defines the cyber manoeuvre as "[...] the application of force to capture, disrupt, deny, degrade, destroy or manipulate computing and information resources in order to achieve a position of advantage in respect to competitors". The main characteristics of a cyber manoeuvre are: speed, operational reach, access and control, dynamic evolution, stealth and limited attribution, rapid concentration, parallel and distributed nature.
Research introduces the following offensive cyber manoeuvre forms -- exploitive manoeuvre (capturing information resources to gain a strategic, operational or tactical competitive advantage), positional manoeuvre (capturing or compromising key physical or logical nodes in the information environment which can then be leveraged during follow-on operations), influencing manoeuvre (using cyber operations to get inside an adversary's decision cycle or even to force that decision cycle though direct or indirect actions). As well as presenting the defensive cyber manoeuvre forms -- perimeter defence \& defence in depth, moving target defence, deceptive defence, and counter attack. 

From a political and strategic stance, Mulvenon \cite{Mulvenon2009} explores the Chinese People's Liberation Army (PLA) conducted computer network operations from the perspective of the scenarios, doctrine, organization, and capabilities. Mulvenon states that defence and pre-emption are the core concepts of PLA and utilize offensive \gls{cno} as an attractive asymmetric weapon against the high-tech adversaries.

With a theoretical interest, Boukerche et.al. \cite{Boukerche2007} explore an agent-based and biological inspired real-time intrusion detection and security model for computer network operations. Research presents a novel intrusion detection model based on artificial immune and mobile agent paradigms for network intrusion detection and response.
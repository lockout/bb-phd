\subsection{Summary and conclusions}
This thesis addresses the lack of public information, understanding, and knowledge on the responsive computer network operation execution within the recently recognized domain of cyber operations. As well as, explores the cyber red teaming applicability to such operation execution, operational asset protection, and required training for the cyber red team. Thesis examines three main closely tied concepts: (1) cyber red team capabilities, novel \gls{tttp}, and engagements in responsive computer network operations; (2) utilization of anomaly detection and cyber deception novel methods for adversary tracking, situational awareness, technical attribution, and red team asset protection; and (3) cyber red team oriented technical exercise design, structure, execution, legal implications, and novel near real-time feedback provision to the training audience.

Chapters \ref{sec:crtopreq} and \ref{sec:crtops} identify and define the building blocks to determine the definition for specialized cyber red team computer network operations. The author addresses the inconsistency and variety of definitions for the concepts of \gls{rcd}, \gls{cno}, and \gls{crt}, and defines a new concept of specialized \gls{crt} responsive \gls{cno} applicable throughout the thesis. As well as, the main characteristics of such operations are estimated. The work in related publications (\ref{pub:firstPub}, \ref{pub:secondPub}, \ref{pub:thirdPub}) is assessed from the perspective of novel \gls{tttp} development applicable to the responsive computer network operation execution by the cyber red team. These techniques and tools are compared and evaluated alongside with other popular and commonly used solutions to elaborate the advantages of the proposed ones. The novel \gls{tttp} are mapped against the identified characteristics of specialized \gls{crt} responsive \gls{cno}, to present their suitability for such operation execution and designated effect delivery.

When considering the specialized cyber red team responsive computer network operations, their extraterritorial nature and offensive capacity has to be acknowledged. For the \gls{crt} to identify, track, and pursue the adversary within the cyberspace, the red team would, to a certain extent, follow the cyber attack execution phases (e.g., reconnaissance, initial foothold, command and control). Such phases, from the defender's perspective, are mapped to a cyber kill-chain, allowing to identify and possibly stopping the incoming attack at every stage of the kill-chain. When assessing the popular and accepted cyber kill-chain methods (e.g., Lockheed Martin, Microsoft, Mandiant/FireEye), it can be seen, that there are areas, where a particular method is stronger or weaker. To address these gaps and emphasize the strengths, a new unified model of ``cyber attack kill-chain'' is proposed by the author. For the \gls{crt} to increase the chances of success, the proper techniques for countering the detection at every phase of the cyber attack kill-chain need to be identified. The proposed cyber red team novel \gls{tttp} are mapped against the cyber attack kill-chain to identify their applicability on countering or supporting it at most of its stages.

Chapter \ref{sec:protection} explores the effects granted by the novel anomaly detection and current cyber deception solutions and approaches. These methods are explored from the perspective on how they can benefit the cyber red team conducted operations. The novel system log file based anomaly detection and cyber deception techniques are evaluated to grant the \gls{crt} the needed scalability, efficiency, ease of automation and management. These approaches fit into the \gls{crt} life-cycle and provide the solutions required for passive defence, syslog analysis, attack surface spoofing, honeypots, cyber decoys, and tarpits. From the delivered effect point of view, the proposed techniques help to detect, deceive, disrupt, degrade, deny, and defend against the adversarial activities.
The responsive \gls{cno} executed by the \gls{crt} would require some form of infrastructure to operate from. This operational infrastructure, depending on the tasks and set objectives, can be used to conduct at least the minimum set of the response related activities, such as, target reconnaissance and vulnerability assessment. The proposed conceptual model for the \gls{crt} operational infrastructure includes the following logical network zones -- cyber operation execution area (location and initial origin of the cyber red team), support area (location, where attacking hosts are deployed with the attack execution supporting resources), staging area (network hosts used to directly interact with the target information system), command and control area (where the \gls{cnc} servers are deployed), and the decoy area (hosting the decoys to where suspicious activity is redirected for further examination). Every particular area has the applicable detection and cyber deception solutions deployed to allow the \gls{crt} maintain the visibility over the deployed technical assets, provide means for adversary tracking in case of a possible counter operation, and contribute to increasing the operational security.

Chapter \ref{sec:exercises} examines the use case of the created and developed technical exercise ``Crossed Swords'' aimed at training the cyber red team in a close to real-life conditions. This exercise integrates all of the listed research work and publications by the author and indirectly serves as the test-bed for novel research conduct and result verification. Besides primarily being a technical exercise, also cyber red team management, leadership, information exchange, and legal issues are tackled, to provide the training experience as close to the real cyber operation execution as possible. One of the corner-stones for the exercise is to explore the cyber kinetic interactions by integrating military units (e.g., special operation forces, military police, and army) into the operational game-play. Such kinetic force game-play is intertwined with the cyber operation to find the synergies, where mutual benefits and cooperation can be identified, emphasized and exercised.
The training audience, assembled into the \gls{crt} and pursuing the assigned mission objectives, practices the existing and novel \gls{tttp} for situational awareness, attribution evidence gathering, target identification, target information system covert infiltration, stealthy activities, and mission objective completion. For this to be successful, a near real-time situational awareness is provided to the participants by the means of novel collection of monitoring, threat detection, and visualization solutions, called the \textit{Frankenstack}. This information gives a timely feedback to the participants to observe the reasons behind the detection of the conducted attacks and allows to identify the needed improvements to increase the level of stealth.

% Research questions and answers to them
\subsection{Answering the research questions}
Each of the research questions is answered by providing the answers to the respective sub-questions.

\textit{[Q1.] What are the specialized cyber red team technical capabilities for responsive computer network operations?}

\textit{[Q1.1.] Which features the techniques, tools, tactics and procedures have to possess to be applicable to the specialized cyber red team responsive computer network operation requirements?}
The identified characteristics of the specialized cyber red team computer network operations include at least the following ones: stealthy and innovative, agile and available, targeted and pervasive, rapid and timely, integrated and coordinated, hybrid and effective, and asymmetric.
The applicable techniques, tools, tactics and procedures have to support these operational characteristics as much as it is possible.
The innovative tools developed based on the suitable techniques have to be at least available on demand, customizable, increase stealth, support asymmetric response, and allow targeted focus of force. This thesis introduces multiple techniques for initial target network access, command and control channel establishment, and impact delivery, which have resulted in the prototyped tools (\textit{tun64}, \textit{nc64}, \textit{Bbuzz}, \textit{iec104inj}, and other proof-of-concept scripts). The suitability of these tools is assessed by comparing them to other relevant and common tools and solutions, and through conducting practical experiments. The results presented in the listed publications (\ref{pub:firstPub}, \ref{pub:secondPub}, \ref{pub:thirdPub}) and summarized in thesis chapter \ref{sec:scrtop} on page \pageref{sec:scrtop}, confirm, that the proposed techniques possess the required features and are applicable to develop suitable tools for specialized cyber red team responsive computer network operation execution, with appropriate tactics and procedures being applied. Furthermore, the proposed techniques and prototyped tools can be also applicable to a wider range of engagements, such as, penetration testing, cyber exercise development and execution, and related cyber operations.

\textit{[Q1.2.] How these proposed techniques, tools, tactics and procedures are suitable to counter the cyber attack kill-chain?}
To provide a more comprehensive cyber attack kill-chain, the existing recognized kill-chains are analysed and consolidated to introduce a new unified model attempting to cover all attack stages as thoroughly as possible. The proposed cyber attack kill-chain, introduced and described in chapter \ref{sec:killchain} on page \pageref{sec:killchain}, presents the following phases: reconnaissance, initial compromise and foothold, command and control, internal reconnaissance, lateral movement, privilege escalation, persistence, asset reconnaissance, and objective completion. For the cyber red team to increase the success of targeting and infiltrating the identified adversarial information systems, suitable \gls{tttp} have to be used to counter possible detection and operation disruption methods at every kill-chain stage. These techniques can be used to support countering all of the cyber attack kill-chain stages either by being directly applied or indirectly supporting other possible methods for countering that specific stage. This is accomplished through granting novel ways for initial compromise and foothold establishment, command and control channel creation, and targeting particular components of the information system. For example, methods for initial foothold can be applicable for gaining further access within lateral movement phase and attempting persistence. Also, established covert command and control channel supports all further activities performed within the target network starting from initial foothold up to objective accomplishment. As well as, targeted vulnerability identification techniques are applicable to support exploitation and privilege escalation attempts.

\textit{[Q2.] How to establish the situational awareness for the cyber red team?}

\textit{[Q2.1.] How system log file analysis and cyber deception approaches are relevant to cyber red team work-flow when executing responsive cyber operations?}
The novel system log file-based correlation and anomaly detection tools in combination with cyber deception solutions, as covered in chapter \ref{sec:protection} on page \pageref{sec:protection}, are applicable to specialized cyber red team responsive computer network operations. For such techniques to be relevant for the inclusion into cyber red team work-flow, they have to possess at least the following characteristics: readily-available and can be obtained by the red team on demand; easily deployable and manageable to lessen the administration overhead and conserve time; flexible and scalable, allowing to be adapted for operational requirements in various environments; highly automated to further reduce the system administration upkeep and human expert involvement; lightweight, permitting fast deployments and having low computing resource consumption requirements; and can be integrated with other technologies already present on the systems. The introduced novel log-based anomaly detection approaches and tested cyber deception solutions (\ref{pub:fourthPub}, \ref{pub:fifthPub}, and \ref{pub:sixthPub}) comply with the presented requirements and are applicable for introduction into cyber red team work-flow.

\textit{[Q2.2.] In what ways such solutions are applicable to situational awareness, red team operational infrastructure protection, and attack technical attribution?}
The chapter \ref{sec:opinfra} on page \pageref{sec:opinfra} introduces and explains the conceptual model for the cyber red team operational infrastructure, where the system log file-based anomaly detection and cyber deception solutions are deployed at various logical network segments. The placement of log analysis and deception tools within those network segments is clarified and their deployment reasoned. The primary goal of such techniques is to aid the cyber red team with providing at least the situational awareness, permit adversary tracking and threat assessment, conduct technical attribution, and protect the cyber red team assets. The proposed techniques and approaches confirm their applicability for cyber red team executed responsive computer network operations.

\textit{[Q3.] How to prepare and train the cyber red team for responsive computer network operation execution?}

\textit{[Q3.1.] What considerations are applicable for training cyber red team as close to the real-life conditions as possible in a technical cyber exercise?}
The chapter \ref{sec:exercises} on page \pageref{sec:exercises} explains and analyses the use case of the cyber red team oriented technical exercise ``Crossed Swords'' series (\ref{pub:tenthPub}).
The identified applicable considerations are at least the following: training environment providing freedom and flexibility to the exercise participants; cyber red team structure design according to the training objectives, mission goals, and cyber operation governance; clear chain-of-command establishment for progress tracking and effective cyber red team management and tasking; sophisticated technical challenges and game-network development, implementing real-life use cases and new technologies (\ref{pub:thirdPub}); appropriate over-arching scenario design to provide the reasoning for the responsive activity immediacy; appropriate technical solution and \gls{tttp} implementation (\ref{pub:firstPub}, \ref{pub:secondPub}, \ref{pub:thirdPub}) allowing to increase the level of stealth and execution speed; near real-time cyber attack detection and feedback provision for the training audience (\ref{pub:seventhPub}); interaction with other operational elements, such as, kinetic forces team, in an integrated game-play to explore the interdependencies, challenges, and benefits; and relevant considerations affecting responsive cyber operation execution, such as, legal ramifications (\ref{pub:eighthPub}). These considerations are confirmed to increase the level of realism and boost the learning experience of the training audience.

\textit{[Q3.2.] What means can be used to assess the cyber red team training objective achievement in near real-time?}
Cyber red team oriented exercise training and mission objectives are aimed at conducting responsive cyber operations to ensure the security and resiliency of protected systems, stop the malicious activities, infiltrate target information system, maintain stealth, and use applicable \gls{tttp}. As such, cyber red team conducted responsive operation would closely follow cyber attack patterns and employ offensive techniques to accomplish the set objectives. 
The requirement for cyber red team executed activity feedback and visualization in near real-time implies high or full automation of various threat detection and attack representation solutions.
Chapter \ref{sec:feedback} on page \pageref{sec:feedback} describes the novel approach used to develop the collection of monitoring, threat detection, data mining, and visualization solutions, called the \textit{Frankenstack}, for the near real-time feedback provision to the cyber red team (\ref{pub:seventhPub}). The learning benefits gained by the use of \textit{Frankenstack} have been verified within the multiple iterations of the ``Crossed Swords'' exercise. As well as, conducting a brief survey among the exercise participants to receive their feedback regarding the exercise and provided situational awareness solutions. Both, the conducted verification and survey, have identified the learning benefits and have been acknowledged by the training audience to boost their experience and skills. Furthermore, the exercise management team has the ability to track the operation execution progress, assess the training objective accomplishment, and evaluate the performance improvement of the training audience throughout the whole exercise in near real-time.


\subsection{Future Work}
The author acknowledges, that while this thesis significantly advances the collective knowledge on cyber red teaming, it can and should be followed up by subsequent research and development.
Author has already started further work on the related topics, such as, applicability and integration of \gls{ai} technologies for the \gls{crt} conducted operations, further development of the ``Crossed Swords'' exercise, cyber red team operational infrastructure automation, and improvement of the \textit{Bbuzz} framework to expand its functionality and apply to network protocol and service vulnerability identification.

To name a few related topics of research to be pursued, such as, cyber command structure for offensive cyber operation execution, cyber weapon development from technical, operational, strategic, and legal perspectives, cyber operation supportive operations, automated exercise game network building by the use of \gls{ai} technologies, and cyber red team structure design approaches and their strengths.
